\documentclass[11pt,a4paper]{article}
\usepackage[utf8]{inputenc}
\usepackage[english]{babel}
\usepackage{graphicx}
\usepackage{multicol}
\usepackage{amsmath}
\usepackage{hyperref}
\usepackage{amsthm}
\theoremstyle{definition}
\newtheorem{definition}{Definition}
\newtheorem{ex}{Example}
\setlength{\columnsep}{1cm}


\begin{document}
\author{\textbf{Ibrahim Abou Elenein}}
\title{\textbf{Fourier Series of CT signals}}
\date {\today}
\maketitle

\textbf{Fourier series} states that, any periodic signal can 
be expressed as summation of sines and cosines 
with one fundamental frequency and infinite number

\section{Fourier Series Representation of CT Periodic Signals}
The exponential form of Fourier series (complex 
representation of continuous time periodic signals) 
\[
    \displaystyle  x(t) = \sum_{k=-\infty}^{\infty} a_k 
    e^{jk\omega_ot}
\]
Whrere: \\
$a_k$ Fourier coefficients.\\
$x(t)$ is periodic with Period $T$ whrere $\omega_o 
= \dfrac{2\pi}{T}$
\[
    x(t) = \dots + a_{-2}e^{-j2\omega_ot}
    + a_{-1}e^{-j\omega_ot} + a_o + a_{1}e^{+j\omega_ot} + \dots
\]

\subsection{Notes}
\begin{itemize}
    \item The term $a_o$ is constant (Dc coefficient)
    \item The terms $a_1  \ \& \  a_{-1}$ both have the fundemntal frequency
        as $\omega_o$ and called \textbf{first harmonic component}
    \item The terms $a_2  \ \& \  a_{0}$ both have the fundemntal frequency
        as $\omega_o$ and called \textbf{seconed harmonic component}
    \item Generalally, The components for $k=-N \ \& \ N$ are called 
        \textbf{Nth order Component}.

\end{itemize}
\section{Obtaining the Fourier Series Coefficients $a_k$}
\[
    \displaystyle x(t) = \sum_{k=-\infty}^{\infty} a_k e^{jk\omega_ot} 
\]
\[
    \displaystyle x(t)e^{-jn\omega_ot} = \sum_{k=-\infty}^{\infty} a_k e^{j(k-n)\omega_ot} \
    \text{integrating over the period}
\]
\[
    \displaystyle \int_0^T x(t)e^{-jn\omega_ot} dt =
    \int_0^T\sum_{k=-\infty}^{\infty} a_k e^{j(k-n)\omega_ot} dt \   a_k 
     \ \text{are time-independent}
\]
\[
    \displaystyle \int_0^T x(t)e^{-jn\omega_ot} dt =
    \sum_{k=-\infty}^{\infty} a_k \int_0^T e^{j(k-n)\omega_ot} dt 
\]
\[
    \displaystyle  \int_0^T e^{j(k-n)\omega_ot} dt = 
    \begin{cases}
        T, & k =n \\
        0, & k \neq n
    \end{cases} \ \text{Integration over one period of cos and sin}
\]

\[
    \underbrace{a_n = \dfrac{1}{T} \int_0^T x(t)e^{-jn\omega_ot} dt }_{\text{Fourier Series Coefficients}}
\]

\section{Fourier series of Periodic CT signal}

\[
    \displaystyle x(t) = \sum_{k=-\infty}^{\infty} a_k e^{jk\omega_ot}
\]
\[
    a_k = \dfrac{1}{T} \int_0^T x(t)e^{-jk\omega_ot} dt \ , \ \omega_o
    = \dfrac{2\pi}{T}
\]
\subsection{Constant Value}
This the average of $x(t)$ over one period.
\[
    a_o = \dfrac{1}{T} \int_T x(t) dt \Rightarrow \text{The area under the curve over one period }
\]
\[
    a_o = \dfrac{\text{Area of one period}}{T}
\]
\section{Famous Signals}
\subsection{DC Signal}
\[
    x(t) = A
\]
\[
    x(t) = \dots + a_{-2}e^{-j2\omega_ot}
    + a_{-1}e^{-j\omega_ot} + a_o + a_{1}e^{+j\omega_ot} + \dots
\]
So by comparing 

\[
    \boxed{\ a_o = A \ \ \& \ \  a_k = 0 \ \text{for} \ k \neq 0}
\]
\subsection{Complex Exponential}
\[
    x(t) = e^{j\omega_ot}
\]
\[
    x(t) = \dots + a_{-2}e^{-j2\omega_ot}
    + a_{-1}e^{-j\omega_ot} + a_o + a_{1}e^{+j\omega_ot} + \dots
\]
So by comparing 
\[
    \boxed{a_1 = 1 \ \ \& \ \ a_k = 0  \ \text{for} \  k \neq 1}
\]
Note if 
\[
    x(t) = 2e^{-3j\omega_ot} \Rightarrow a_{-3} = 2 \ \ \& \ \ a_k = 0  \ \text{for} \ k \neq -3
\]
\subsection{Sinusoidals}

\[
    x(t) = \cos(\omega_ot) = \underbrace{\dfrac{1}{2} e^{j\omega_ot} + \dfrac{1}{2} e^{-j\omega_ot}}_{\text{Euler's rule}}
\]
\[
    x(t) = \dots + a_{-2}e^{-j2\omega_ot}
    + a_{-1}e^{-j\omega_ot} + a_o + a_{1}e^{+j\omega_ot} + \dots
\]
So by comparing 
\[
    \boxed{a_1 = a_{-1} = \frac{1}{2} \ \ \& \ \ a_k = 0  \ \
    \text{for} \ \ k \neq 1, -1 }
\]
\end{document}
