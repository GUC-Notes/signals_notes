\documentclass[11pt,a4paper]{article}
\usepackage[utf8]{inputenc}
\usepackage[english]{babel}
\usepackage{graphicx}
\usepackage{multicol}
\usepackage{amsmath}
\usepackage{hyperref}
\usepackage{amsthm}
\theoremstyle{definition}
\newtheorem{definition}{Definition}
\newtheorem{ex}{Example}
\setlength{\columnsep}{1cm}

 
\begin{document}
\author{Ibrahim Abou Elenein}
\title{Unit Step and Unit Impulse}
\date {\today}
\maketitle
\section{Discrete Unit Impulse }
\[
    \delta(n) = \begin{cases}
        1, &n = 0 \\
        0, & n \neq 0 
    \end{cases}
\]

\section{Discrete Unit Step}
\[
    u(n) = \begin{cases}
        1, &n \geq 0 \\
        0, & n < 0 
    \end{cases}
\]
\section{Rectangler Signal}
\[
    x(n) = \begin{cases}
        1, & 0 \leq n \leq N \\
        0, & otherwise
    \end{cases}
\]
\section{Continuous Unit Step}
\[
    u(t) = \begin{cases}
        1, &t > 0 \\
        0, & t < 0 
    \end{cases}
\]
\section{Continuous Unit Impulse}
    \[
        \delta(t) = 0 \ for \  t \neq 0 
    \]
\section{Relation between Unit Impulse and Unit Step}
\[
    \delta(n) = u(n) - u(n-1)
\]
\[
    \delta(t) = \frac{d}{dt} [u(t)] 
\]
\[
    \displaystyle u(n) = \sum_{k=0}^{\infty} \delta(n-k)
\]
\[
    \displaystyle u(n) = \sum_{m=-\infty}^{n} \delta(m)
\]
\[
    \displaystyle u(t) = \int_{-\infty}^{t} \delta(\tau) d\tau
\]

\section{Notes}
\[
    \int_{-\infty}^{\infty} \delta(\tau)d\tau = 1
\]
\[
    x(n) = \sum_{k=-\infty}^{\infty} x(k) \delta(n-k)
\]
\[
    \displaystyle \int_{t_o - \epsilon}^{t_o + \epsilon}  
    f(t)\delta(t-t_o)dt = f(t_o) \forall \ \ \ \epsilon > 0
\]
Example
\[
    \int_0^2 sin(\pi t) \delta(t-2.5) dt  
\]
at $ t = 2.5 $ this  value are not in the interval [0,2] so it's zero



Example
\[
    \int_{-4}^7 sin(\frac{\pi t}{2}) \delta(t-1) dt  
\]
the value $ t = 1 $ is in the interval so 
\[
    \int_{-4}^7 sin(\frac{\pi t}{2}) \delta(t-1) dt =
    f(1) = \sin(\frac{\pi \times 1}{2}) = \sin(\frac{\pi}{2}) = 1
\]
\end{document}
