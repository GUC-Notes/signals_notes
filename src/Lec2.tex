\documentclass[11pt,a4paper]{article}
\usepackage[utf8]{inputenc}
\usepackage[english]{babel}
\usepackage{graphicx}
\usepackage{multicol}
\usepackage{amsmath}
\usepackage{hyperref}
\usepackage{amsthm}
\theoremstyle{definition}
\newtheorem{definition}{Definition}
\newtheorem{ex}{Example}
\setlength{\columnsep}{1cm}

 
\begin{document}
\author{Ibrahim Abou Elenein}
\title{Lecture 2 Summary}
\date {\today}
\maketitle

\section{Periodicity}
\subsection{General}
\subsubsection{Continuous} 
    
if $ x(t) = x(t+T) \Longrightarrow x(t)$ is Periodic 
    
    
\subsubsection{Discrete}
    
if $ x[n] = x[n+N] \Longrightarrow x[n]$ is Periodic  
\subsection{Special Cases}
\subsubsection{Continuous}
Complex exponential is always periodic with period $ T_o = \frac{2\pi}{\omega _ o}$ \par

\begin{ex}
\[
    e^{j\omega_o t} \ , \  \cos \omega_o t \ ,  \ \sin \omega _o t
     \ \ \ \ \ \ \ T_o = \frac{2\pi}{\omega _o}
\]
\end{ex}

\subsubsection{Discrete}
Complex exponential may be periodic if $\frac{\omega _o}{2\pi}$ = rational number = $\frac{m}{N}$ where N is the period  

\begin{ex}
    \[
        e ^{j\omega_o n} \ , \ \cos(\omega _o n) \ , \ \sin(\omega _o n)
    \]
\end{ex}

    if $\frac{\omega_o}{2\pi} = \frac{m}{N}$ where N = Period 
    
\subsection{Examples}

\begin{itemize}
    \item  $x(t) = j e^{j10t}$ \ \ \ \ \ \ 
    Periodic with $T_o = \frac{2\pi}{\omega_o} = \frac{\pi}{5}$
   
   \item $ x[n] = e^{j\pi n}$  \ \ \ \ \ $\frac{\omega_o}{2\pi} = \frac{1}{2}$ \ \  \ $N = 2$ 
   
   \item $x[n] = \cos(\frac{2\pi}{4}n)$  \ \ $\frac{\omega_o}{2\pi} = 
        \frac{1}{4}$
    \item $x[n] = 3 e ^{j\frac{3\pi}{5}(n + \frac{1}{2})} = 3 e^{j\frac{3\pi}{5}(0.5)}. e^{j \frac{3\pi}{5}n}$ \  \ 
    $\frac{\omega_o}{2\pi} = \frac{3}{10}$
\end{itemize}
\subsection{Notes}
\begin{itemize}
    \item constant functions is periodic with arbitrary period
    \item Periodic $\pm$ aperiodic  = aperiodic 
    \item periodic x  aperiodic = aperiodic 
    \item periodic $\pm$  periodic = periodic  \ \ if $\frac{\omega _o1}{\omega_o 2} =$ rational \\
    
    The Period of the sum of two periodic signals is the lowest common
    multiple LCM of the individual periods. \\
    
    The frequency of the sum of two periodic signal is the greatest common divisor GCD of the individual frequencies. 

\end{itemize}
\subsubsection{Examples}

\begin{enumerate}
    \item $x(t) = e^{(j-1)t} = e^{jt}(complex \ \ exp) * e^{-t}(real  \  \ exp) $ = \\  periodic x aperiodic  $\longrightarrow x(t)$ is aperiodic 
    
    \item $x(t) = \cos(\pi t) + \sin(2t)$  \ \ $\frac{\omega_{o1}}{\omega_{o2}} = \frac{\pi}{2} $ is not rational s $x(t)$ is not periodic 
\end{enumerate}

\end{document}
